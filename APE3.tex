\newape

\begin{solution}
  \begin{enumerate}
    \item
      To have $m^{ed} \equiv m \pmod{N}$,
      it is sufficient to have $ed \equiv 1 \pmod{\varphi(N)}$.
      To have this, we need $e$ relatively prime to $\varphi(N)$.

      $\varphi(N) = (7-1)(11-1) = 60 = 2^235$
      so the smallest $e$ we can take is $7$.
    \item
      To find $d$ such that $ed \equiv 1 \pmod{60}$ we
      can use the extended Euclidean algorithm

      \begin{align*}
        60 & = 8 \cdot 7 + 4\\
        7 & = 1 \cdot 4 + 3\\
        4 & = 1 \cdot 3 + 1
      \end{align*}
      so

      \begin{align*}
        1 & = 4 - 1 \cdot 3\\
          & = 2 \cdot 4 - 1 \cdot 7\\
          & = 2 \cdot 60 - 17 \cdot 7.
      \end{align*}
      We can take $-17 \equiv 43 \pmod{60}$ for $d$.
    \item
      We have

      \begin{align*}
        c & \equiv 75^7 \pmod{77}\\
          & \equiv (-2)^7 \pmod{77}\\
          & \equiv -128 \pmod{77}\\
          & \equiv 26 \pmod{77}.
      \end{align*}
    \item
      Let's just verify it modulo $7$ and $11$ which is equivalent
      according to the CRT

      \begin{align*}
        c^{43} & \equiv 26^{43} \pmod{7}\\
               & \equiv (-2)^{7 \cdot \varphi(7) + 1} \pmod{7}\\
               & \equiv -2 \pmod{7}\\
               & \equiv 75 \pmod{7}
      \end{align*}
      and

      \begin{align*}
        c^{43} & \equiv 26^{43} \pmod{11}\\
               & \equiv 4^{4 \cdot \varphi(11) + 3} \pmod{11}\\
               & \equiv 4^{3} \pmod{11}\\
               & \equiv 64 \pmod{11}\\
               & \equiv -2 \pmod{11}\\
               & \equiv 75 \pmod{11}.
      \end{align*}
  \end{enumerate}
\end{solution}

\begin{solution}
  \begin{enumerate}
    \item
      \begin{enumerate}
        \item
          $\epsilon_7$ should be a multiple of $11$ so there is a $y$ such
          that $\epsilon_7 = 11y$.
          So we have

          \begin{align*}
            11y & \equiv 1 \pmod{7}\\
              y & \equiv (11)^{-1} \pmod{7}\\
                & \equiv 2 \pmod{7}
          \end{align*}
          which gives $\epsilon_7 \equiv 22 \pmod{77}$.
        \item
          $\epsilon_{11}$ should be a multiple of $7$ so there is a $y$ such
          that $\epsilon_{11} = 7y$.
          So we have

          \begin{align*}
            7y & \equiv 1 \pmod{11}\\
             y & \equiv 7^{-1} \pmod{11}\\
               & \equiv 8 \pmod{11}
          \end{align*}
          which gives $\epsilon_{11} \equiv 56 \pmod{77}$.
        \item
          By the superposition principle
          \begin{align*}
            5\epsilon_7 + 9\epsilon_{11} & \equiv 5 \pmod{7}\\
            5\epsilon_7 + 9\epsilon_{11} & \equiv 9 \pmod{11}
          \end{align*}
          so $x \equiv 5 \cdot 22 + 9 \cdot 56 \equiv 75 \pmod{77}$.
        \item
          We can take $x_p\epsilon_7 + x_q\epsilon_{11}$
          since by the superposition principle
          \begin{align*}
            x_p\epsilon_7 + x_q\epsilon_{11} & \equiv x_p \cdot 1 + x_q \cdot 0 \pmod{7}\\
                                             & \equiv x_p \pmod{7}\\
            x_p\epsilon_7 + x_q\epsilon_{11} & \equiv x_p \cdot 0 + x_q \cdot 1 \pmod{11}\\
                                             & \equiv x_q \pmod{11}.
          \end{align*}
        \item
          We can know that $\epsilon_p = qy$ for some $y$
          So we have

          \begin{align*}
            qy & \equiv 1 \pmod{p}\\
             y & \equiv q^{-1} \pmod{p}\\
          \end{align*}
          which gives $\epsilon_7 \equiv qq^{-1} \pmod{pq}$
          where the inverse is taken modulo $p$.

          This is the same for $\epsilon_q$.
        \item
          \[ \psi^{-1} : \mathbb{Z}_p \times \mathbb{Z}_q \to \mathbb{Z}_n; (x_p, x_q) \to x_p\epsilon_p + x_q\epsilon_q. \]
      \end{enumerate}
  \end{enumerate}
\end{solution}

\begin{solution}
  \begin{enumerate}
    \item
      It is a subset of $\mathbb{Z}_p^*$ since $0$
      cannot be the square of an element in $\mathbb{Z}_p^*$.

      It is a group since if $x_1$ and $x_2$ are quadratic residues,
      there exists $y_1,y_2$ such that
      \begin{align*}
        y_1^2 & \equiv x_1 \pmod{p}\\
        y_2^2 & \equiv x_2 \pmod{p}
      \end{align*}
      so
      \[ (y_1y_2)^2 \equiv x_1x_2 \pmod{p} \]
      and $x_1x_2$ is also a quadratic residue.
  \end{enumerate}
\end{solution}

\nosolution
\nosolution
\nosolution
