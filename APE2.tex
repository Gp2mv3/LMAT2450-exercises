\newape

\begin{solution}
  \begin{itemize}
    \item
      Let $\tilde{\Pi} = \langle \tilde{\Gen}, \tilde{\Mac}, \tilde{\Vrfy} \rangle$, defined as:
      \begin{itemize}
        \item $\tilde{\Gen}$: chooses a random $f$.
        \item $\tilde{\Mac}$: on input $m$, outputs $f(m)$.
        \item $\tilde{\Vrfy}$: on input $(m,t)$, outputs $1$ iff $f(m) = t$.
      \end{itemize}

      Let's analyse the maximum value of $\Pr[\MacForge_{\A, \tilde{\Pi}}(n) = 1]$ for an adversary $\mathcal{A}$.
      If after $q$ different queries (it gains no info doing the same query twice),
      $m_1, \ldots, m_q$, $\A$ outputs $(m, t)$, what are its chances of success ?
      Let $f:\{0,1\}^n \to \{0,1\}^n$.
      There are $(2^n)^{2^n}$ different $f$ and we pick a random one uniformly.
      However, there are only $(2^n)^{2^n-q}$ experiments such that $\A$ have received $(m_i,t_i)$ for $i = 1, \ldots, q$ because
      there are $(2^n)^{2^n-q}$ $f$ such that $f(m_i) = t_i$ for $i = 1, \ldots, q$.
      We could be in any of them.
      Among them, only $(2^n)^{2^n-(q+1)}$ are such that $f(m) = t$.
      Since $f$ is selected uniformly, we have

      \begin{align*}
        \Pr[\MacForge_{\A, \tilde{\Pi}}(n) = 1]
        & = \Pr[f(m) = t | f(m_i) = t_i, \forall m = 1, \ldots, q]\\
        & = \frac{\Pr[f(m) = t, f(m_i) = t_i, \forall m = 1, \ldots, q]}{\Pr[f(m_i) = t_i, \forall m = 1, \ldots, q]}\\
        & = \frac{\frac{(2^n)^{2^n-(q+1)}}{(2^n)^{2^n}}}{\frac{(2^n)^{2^n-q}}{(2^n)^{2^n}}}\\
        & = \frac{(2^n)^{2^n-(q+1)}}{(2^n)^{2^n-q}}\\
        & = \frac{1}{2^n}.
      \end{align*}
      A shortcut would have been to argued that since, $f(m)$ is independent of the $f(m_i)$ so

      \begin{align*}
        \Pr[\MacForge_{\A, \tilde{\Pi}}(n) = 1]
        & = \Pr[f(m) = t | f(m_i) = t_i, \forall m = 1, \ldots, q]\\
        & = \Pr[f(m) = t]\\
        & = \frac{(2^n)^{2^n-1}}{(2^n)^{2^n}}\\
        & = \frac{1}{2^n}.
      \end{align*}

      It is quite surprising that instead of a upper bound
      on $\Pr[\MacForge_{\A, \tilde{\Pi}}(n) = 1]$
      depending on $\A$ (and reached for $\A$ super smart),
      it is actually independent of $\A$.
    \item
      Let's now suppose that we have an adversary $\A$
      that win with non-neglishible probability against a PRF $F$
      and show that under this assumption we can build
      a distinguisher $\D$ for $F$.

      $\D$ will simply take a function $g$ as input
      and run $\A$ using $g$ to create the tags.
      He has $g$ so he can see if $\A$ wins or lose.
      If $\A$ wins he outputs $1$ otherwise it outputs $0$.

      We know that if $g$ is a pseudo random function,
      $\Pr[\MacForge_{\A, \Pi_g} = 1] = \frac{1}{2^n}$
      and if $g$ is a PRF
      $\Pr[\MacForge_{\A, \Pi_g} = 1] = \eta(n)$
      where $\eta$ is nonnegligible.
      We have therefore
      \[
        |\Pr[\D^{F_k}(1^n) = 1] - \Pr[\D^{f}(1^n) = 1]|
        = \left|\eta(n) - \frac{1}{2^n}\right|
      \]
      which is non-negligible.

      % Not sure it works
%    \item
%      Another simpler solution is possible. Using the first Hint, we can say that if $F_k$ is a PRF, it has a maximum of $2^n$ possible outputs
%      where a truly random function has exactly $2^n$ outputs. So if we suppose $\Pi$ secure with a PRF, then $\tilde{\Pi}$ is also secure because
%      $\epsilon_{\tilde{\Pi}}  = \frac{1}{2^n} \geq \epsilon_{\Pi}$. We then can play the PRF game to prove the security of $\Pi$ with the second hint.
  \end{itemize}

\end{solution}

\begin{solution}
  If we have the tag of $p$, which is (let's consider that $k$ and $p$ are $l$ bits long for simplicity)
  \[ t_p = H^s(k\|p) = h^s(h^s(h^s(h^s(IV \| k) \| p)), l) \]
  we can find the tag of $p\|l\|w$ (where $w$ is $l$ bits long for simplicity)
  without knowing $k$ since we know $h^s$.
  It is
  \begin{align*}
    H^s(k\|p\||p|\|w)
    & = h^s(h^s(h^s(h^s(h^s(h^s(IV \| k) \| p)) \| l) \| w) \| 3l)\\
    & = h^s(h^s(H^s(k \| p) \| w) \| 3l)\\
    & = h^s(h^s(t_p \| w) \| 3l)
  \end{align*}
  Since $p\|l\|w \neq p$, this gives us an existential forgery.
\end{solution}

\begin{solution}
  From $L_0\|R_0$, we get $L_2\|R_2$
  where

  \begin{align*}
    L_2 & = L_0 \xor F_k(R_0)\\
    R_2 & = R_0 \xor F_k(L_0 \xor F_k(R_0)).
  \end{align*}
  When a distinguisher $\D$ sends $L_0\|R_0$ and receives
  $L_2\|R_2$ we can
  compute

  \begin{align*}
    L_2' & = L_0 \xor L_2\\
         & = L_0 \xor (L_0 \xor F_k(R_0))\\
         & = F_k(R_0)\\
    R_2' & = R_0 \xor R_2\\
         & = R_0 \xor (R_0 \xor F_k(L_0 \xor F_k(R_0)))\\
         & = F_k(L_0 \xor F_k(R_0)).
  \end{align*}
  The problem is that $L_2'$ does not depend on $L_0$.

  He can therefore send another request with the same $R_0$
  and a different $L_0$.
  If it gets the same $L_2'$,
  It has only 1 chance over $2^l$ (where $l$ is the number of bits of $L$) to receive the same $L_2'$ for a random permutation.

  If $\D$ decide to output 1 iff $L_2'$ is the same for the two request we have
  \[ |\Pr[\D^{F_k}(1^n) = 1] -\Pr[\D^f(1^n)]| = 1 - \frac{1}{2^n} \]
  which is clearly not negligible.
\end{solution}

\begin{solution}
  Let say we have to split the message into $m$ messages of $n$ bits.
  Choosing $m_0 = M_0 \| M_0 \| \cdots \| M_0$ and $m_1 = M_1 \| M_2 \| \cdots \| M_m$ with $M_i \neq M_j$ for $i \neq j$,
  if $b = 0$, we will get $c = C_0 \| C_0 \| \cdots \| C_0$ for some $C_0 \in \C$ and $c = C_1 \| C_2 \| \cdots \| C_m$ for some $C_i \in \C$.

  An adversary $\A$ can output $b = 0$ iff all the $C_i$s are equals.
  We have
  \[ \Pr[\PrivKeav_{\A,\text{ECB}}(nm)] = \frac{1}{2} + \frac{1}{2} = 1 \]
  since the $C_i$ cannot be equal if $b = 1$ since we use a PRP.
  If two different $M_i$ were encrypted as same $C_i$, decryption wouldn't be possible.
\end{solution}

\begin{solution}
  We will give 2 collisions for $f_1$ and $f_2$.

  We know that $E(k, \cdot)$ is surjective because it must be injective and $\M = \C$.
  Therefore the decryption exists for all $c \in \C$ ! We will use it for the second collision of $f_1$.

  For $f_1$, we have the 2 following collisions

  \begin{align*}
    f_1(D(E(y,x),y), E(y,x))
    & = E(E(y,x), D(E(y,x), y)) \xor E(y,x)\\
    & = y \xor E(y,x)\\
    & = E(y,x) \xor y\\
    & = f_1(x, y)\\
    f_1(D(0, E(y,x) \xor y), 0)
    & = E(0, D(0, E(y, x) \xor y)) \xor 0\\
    & = E(y, x) \xor y\\
    & = f_1(x, y).
  \end{align*}
  and for $f_2$ we have

  \begin{align*}
    f_2(x, E(x,x)) & = E(x,x) \xor E(x,x)\\
                   & = 0 & \forall x \in \{0,1\}^l\\
    f_2(y, E(x,x)) & = E(y,y) \xor E(x,x)\\
                   & = E(x,x) \xor E(y,y)\\
                   & = f_2(x, E(y,y)).
  \end{align*}
\end{solution}

\begin{solution}
  \begin{itemize}
    \item
      Let's show that from a collision of $H_4$, we generate a collision for $H_2$
      which prove that $H_4$ is collision resistant since $H_2$ is so.
      Let's suppose that we have $x_1\|x_2 \neq y_1\|y_2$ are such that $H_4(x_1\|x_2) = H_4(y_1\|y_2)$.
      \begin{itemize}
        \item
          If $H_2(x_1) \| H_2(x_1 \xor x_2) \neq H_2(y_1) \| H_2(y_1 \xor y_2)$,
          we have a colision for $H_2$ since their image by $H_2$ is identical.
        \item
          If $H_2(x_1) \| H_2(x_1 \xor x_2) = H_2(y_1) \| H_2(y_1 \xor y_2)$,
          we have $H_2(x_1) = H_2(y_1)$ \emph{and} $H_2(x_1 \xor x_2) = H_2(y_1 \xor y_2)$.
          \begin{itemize}
            \item If $x_1 \neq y_1$, we have a collision for $x_2$ since $H_2(x_1) = H_2(y_1)$.
            \item If $x_1 = y_1$, then $x_2 \neq y_2$ since $x_1\|x_2 \neq y_1\|y_2$.
              Therefore $x_1 \xor x_2 \neq y_1 \xor y_2$ and we have collision on $H_2$.
          \end{itemize}
      \end{itemize}
    \item
      $H_6$ is not collision resistant since $H_6(x_1\|x_2\|x_3) = H_6((x_1 \xor w)\|(x_2 \xor w)\|(x_3 \xor w))$
      for all $w$.
  \end{itemize}
\end{solution}

\begin{solution}
  \begin{enumerate}
    \item Encrypt all the DVDs with $K_\text{root}$ !
    \item At the beginning of the DVD, encrypt a random key $K$ twice (using $K_2$ and then $K_3$): $Enc_{K_2}(K)$ and $Enc_{K_3}(K)$.

      We then encrypt all the content $M$ of the DVD using $K$. (We suppose $K_1$ is $K_{\text{root}}$)
      $K_2$ and $K_3$ are the keys associated to the 2 childs of the root.
    \item
      For every node $i$ on the path from the root to $r$, we must add an encryption of $K$ with the key of its child that is not in the path.
      For example, if $r = 10$ and $n = 16$, we must include the bold keys,
      so we will include the encryption of $K_7$ in a DVD which is quite something.
      \begin{center}
        \Tree [.{$K_{\text{root}} = K_1$}
          [.{$\mathbf{K_2}$}
            [.{$K_4$}
              [.{$K_8$}
                [.{0} ]
                [.{1} ]
              ]
              [.{$K_9$}
                [.{2} ]
                [.{3} ]
              ]
            ]
            [.{$K_5$}
              [.{$K_{10}$}
                [.{4} ]
                [.{5} ]
              ]
              [.{$K_{11}$}
                [.{6} ]
                [.{7} ]
              ]
            ]
          ]
          [.{$K_3$}
            [.{$K_6$}
              [.{$\mathbf{K_{12}}$}
                [.{8} ]
                [.{9} ]
              ]
              [.{$K_{13}$}
                [.{10} ]
                [.{11} ]
              ]
            ]
            [.{$\mathbf{K_7}$}
              [.{$K_{14}$}
                [.{12} ]
                [.{13} ]
              ]
              [.{$K_{15}$}
                [.{14} ]
                [.{15} ]
              ]
            ]
          ]
        ]
      \end{center}
  \end{enumerate}
\end{solution}

\begin{solution}
  \begin{enumerate}
    \item TODO
    \item $\A'$ should pick a random $r$ and send to its oracle the $d$ queries $r\|l\|i\|m_i$ for $i = 1, \ldots, d$.
      and output $\langle r, t_1, \ldots, t_d \rangle$ where $t_i$ is the answer of the oracle to its $i$th query.
    \item For $\Pi$ a forgery is $(m, \langle r, t_1, \ldots, t_d \rangle)$ where $m$ is not one of its previous query
      and $t_i = \Mac_k(r\|l\|i\|m_i)$ for $i = 1, \ldots, d$ where $l$ is the length of $m$ and $m = m_1\| \cdots \|m_d$.
      \begin{itemize}
        \item
          If none of its previous query has the same $r$ and $l$, $r\|l\|1\|m_1$ cannot be a query made by $\A'$ and
          $(r\|l\|1\|m_1, t_1)$ is an existential forgery for $\Pi'$ that $\A'$ can output.
        \item
          If 2 previous queries have both $r$ and $l$, then we do not necessarily have an existential forgery.
          However, since $r$ is picked at random ($\A'$ could cheat and make sure that the same $r$ is not picked twice)
          the probability (birthday paradox) of this to happen (if all $m$ have the same $l$ which is the worst case) is approximatively
          $\frac{q(n)^2}{2 \cdot 2^n}$ where $q(n)$ is the number of queries make by $\A$.
        \item
          If one unique previous query $m^j$ has this $r$ and $l$, since $m$ is not one of the previous query, there must be $i$
          such that $m_i \neq m_i^j$.
          We know therefore that $r\|l\|j\|m_j$ has never been queried by $\A'$ so $(r\|l\|j\|m_j, t_j)$ is an existential forgery for $\Pi$.
      \end{itemize}
    \item In conclusion, we have
      \begin{align*}
        \Pr[\MacForge_{\A,\Pi}(n) = 1]
        & \leq \Pr[\MacForge_{\A',\Pi'}(n) = 1] + \frac{q(n)}{2^{n+1}}\\
        \epsilon(n)
        & \leq \epsilon'(n) + \frac{q(n)}{2^{n+1}}
      \end{align*}
      so since $\epsilon'(n)$ and $\frac{q(n)}{2^{n+1}}$ are negligible,
      $\epsilon(n)$ is negligible.
  \end{enumerate}
\end{solution}

\begin{solution}
  We build $H(m) = H_0(m)\|H_1(m)$.
  If we have $m_1 \neq m_2$ such that $H(m_1) = H(m_2)$ then $H_0(m_1) = H_0(m_2)$ and
  $H_1(m_1) = H_1(m_2)$ so the collision resistant hash function has a collision whichever it is.
  However, $H$ is no more a compression function and we cannot use Merkle-Damg\aa{}rd.

  The input of $H$ therefore cannot have arbitrary length but its input is twice the length of the output of $H_0$ and $H_1$
  so it is twice the size of a tag.

  The output of $\Mac_0$ and $\Mac_1$ are the size of a tag so we can use the tag
  $H(\Mac_0(k,m)\|\Mac_1(k,m))$ for our Hash-MAC scheme.
  If we are able to output an existential forgery $(m, t)$,
  since $H$ is collision resistant, that means that we have found
  $\Mac_0(k,m)\|\Mac_1(k,m)$ and therefore we have found an existantial forgery for both
  $\Mac_0$ \emph{and} $\Mac_1$ which is absurd since one of them is believed to be unforgeable.

  Our Hash-MAC scheme is therefore unforgeable.
\end{solution}

\begin{solution}
  \begin{enumerate}
    \item When player $\ell$ receives $(i,c_0,c_1)$ he can compute its binary representation $\ell = \sum_{j=0}^{n-1} b_j2^j$
      and then computes
      \begin{align*}
        D(k_{i,b_i}, c_{b_i})
        & = D(k_{i,b_i}, E(k_{i,b_i}, m))\\
        & = m.
      \end{align*}
    \item The content provider can recover $b_j$ by feeding $P$ with $(i,E(k_{j,0},0),E(k_{j,1},1))$.
      He will then listen to what $P$ broadcast.
      $b_j$ will be what he will hear.
    \item $i \neq j$ and $i \xor j$ is not a power of two.
      That means that $i$ and $j$ differ for at least 2 bits.
      For one of them they will use the key of $i$ and for the other they will use the key of $j$.
      For example, if $i = 101_2$ and $j = 110_2$, they will use $k_{2,1}$, $k_{1,0}$ and $k_{0,0}$.
      That way the content provider will think that the index is $100_2$.
  \end{enumerate}
\end{solution}

\begin{solution}
%   This scheme is not secure:
%
%   We can define the cipher as $Enc_k(m_1, m_2) = C = c_1\| c_2 \| c_3 = p_k(IV\xor m_1) \| p_k(m_2 \xor p_k(IV\xor m_1)) \| p_k(m_3 \xor p_k(m_2 \xor p_k(IV\xor m_1)))$.
%   $IV = 0^n$ so we know that $c_1 = p_k(m_1)$, using the inverse of $p_k$ (assumed known by the exercice),
%   we can retrieve $m_1$. We can also retrieve $m_2 \xor c_1$ using the same method.
%   The last block (encrypthing $r$) is not really interresting as we've already retrieved
%   the messages. The solution to protect the scheme is to put the $r$ block at first.
%
% \paragraph{Note: }
  Since $r$ is not outputted by the encryption, this is not an encryption scheme.
  Indeed, $\Enc$ is not injective so we cannot define $\Dec$.
  For example, the decryption of $p_k(0)$ could be any $m_1\|m_2$ because
  if $r = p_k(p_k(m_1) \xor m_2)$,

  \begin{align*}
    \Enc_k(m_1\|m_2) & = p_k(p_k(p_k(m_1) \xor m_2) \xor p_k(p_k(m_1) \xor m_2))\\
                     & = p_k(0).
  \end{align*}

  I guess this solution is not the expected one so if someone could tell me what I should do, that would be nice.

\end{solution}

\begin{solution}
  \begin{enumerate}
    \item
      The problem with a MAC is that every party that can verify a MAC can also build it since verifying a MAC consist
      in recomputing the tag and comparing it the the received tag.
      All $B_i$ therefore need to have the key $k$ and can therefore send valid packets to other $B_j$.
    \item
      For $B_i$ to build all the MACs for $B_j$, he needs $S_i \subseteq S_j$.
      We therefore need that there is no $i \neq j$ such that $S_i \subseteq S_j$.
    \item We can simply take all $S_i$ having 2 elements.
      $S_i \subseteq S_j$ simplifies therefore in $S_i \neq S_j$.
      There are ${5 \choose 2} = \frac{5!}{2! \cdot 3!} = 10$ possible such sets.
      %Sure ? Because I think in this case B_k would be allowed to sign packets to other users haveing k_k in his set. No ?
  \end{enumerate}
\end{solution}
